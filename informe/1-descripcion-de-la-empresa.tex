
\section{Descripción de la empresa}

PedidosNow es una empresa argentina que provee un servicio de seguimiento y gestión de pedidos y entregas de comida a domicilio. Fue fundada en 2010 y desde entonces se ha expandido a más de 20 países de Latinoamérica.

La empresa cuenta con más de 30 millones de usuarios, más de 15000 cadenas de restaurantes y alrededor de 300.000 repartidores. Realiza más de 23 millones de pedidos al año, obteniendo una ganancia anual de \$3.500 millones de dólares. Cotiza en bolsa con una valuación de \$35.000 millones de dólares.

\subsection{Actividades que realiza la empresa}

La actividad principal de la empresa es conectar a los usuarios, que desean hacer pedidos de comida a domicilio, con sus más de 15.000 restaurantes. PedidosNow se encarga de procesar el pago y el envío a domicilio, de tal forma que el usuario tenga una experiencia sencilla y satisfactoria.

Otorga a los usuarios la posibilidad de observar y comparar los distintos productos ofrecidos por los restaurantes y los precios de los mismos. Una vez que el usuario ha decidido su compra, la empresa se encarga de realizar el pago al restaurante, quedándose con una comisión, enviar al mismo la orden del cliente, y asignar el pedido a un repartidor para su reparto. Ofrece al usuario la capacidad para realizar el seguimiento de su pedido y de observar sus pedidos pasados.

Ofrece a los restaurantes afiliados la capacidad de cargar, dar de baja y modificar sus sucursales, horarios de atención, y sus productos ofrecidos junto con sus precios.

Asigna a los repartidores los pedidos a realizar, indicándoles las direcciones de partida y llegada, detalles de la orden, y algunos datos del cliente. Cuando el repartidor completa un envío de forma exitosa, se le otorga una comisión.


\subsection{Estructura de la empresa}

La empresa está organizada en las siguientes áreas:

\begin{itemize}
    \item Gerencia
    
    Consiste de los ejecutivos de alto nivel, incluyendo al CEO, CFO, COO, y CTO. Decide los objetivos y metas de la empresa, toma las decisiones de mayor responsabilidad, y realiza acuerdos de gran envergadura con otras empresas.

    \item Logística
    
    Su principal tarea es manejar los acuerdos con los restaurantes. También es responsable de mantener la cantidad adecuada de repartidores disponible y la correcta asignación de los pedidos.
    
    \item Tecnología
    
    Encargada principalmente de mantener el software y la infraestructura de la empresa. También está encargada de la ciberseguridad y del análisis de datos.

    \item Marketing
    
    Responsable de las campañas de marketing, SEO, retención de usuarios, promociones, la identidad de la marca, y el uso de redes sociales.

    \item Recursos Humanos
    
    Se encarga de la contratación de personal, pago de sueldos y beneficios.

    \item Finanzas y Legal
    
    Se encarga de llevar la contabilidad de la empresa y cumplir con sus obligaciones financieras y legales.

    \item Soporte
    
    Provee soporte a clientes, restaurantes y repartidores respecto al uso de la plataforma, reembolsos, problemas con los envíos, etc.
\end{itemize}

\subsection{Organigrama de la empresa}

\dots

\subsection{Otra información relevante}

\dots
