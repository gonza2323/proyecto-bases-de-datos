
\section{Descripción del sistema a desarrollar}
El sistema a desarrollar consiste en una plataforma de pedidos de comida a domicilio, accesible tanto desde la web como desde aplicaciones móviles. Su objetivo principal es ofrecer una experiencia fluida, eficiente y segura para conectar a clientes, restaurantes y repartidores, permitiendo la exploración de menús, la realización de pedidos, el procesamiento de pagos, y la gestión de entregas de forma integrada y automatizada.

La empresa busca una solución moderna, capaz de manejar un alto volumen de usuarios, brindar herramientas analíticas a los restaurantes, y mejorar la eficiencia operativa general, reduciendo tiempos de entrega y optimizando el uso de recursos.

El sistema contempla tres tipos principales de usuarios: clientes, repartidores y restaurantes, cada uno con sus propias necesidades y funcionalidades específicas. Entre los principales requisitos se incluyen registro y autenticación, visualización de restaurantes por zona, gestión de pedidos y pagos, interfaces específicas para cada tipo de usuario, y generación de reportes para los comercios. Además, se establecen altos estándares de usabilidad, rendimiento, disponibilidad, seguridad y mantenibilidad.

Aunque se implementará gran parte del modelo de datos de las funcionalidades planificadas, ciertas características quedarán fuera del alcance de esta versión inicial, como el registro automático repartidores y clientes, seguimiento en tiempo real, procesamiento de pagos con tarjeta y el sistema de atención al cliente integrado. Estas funcionalidades podrán ser consideradas en futuras fases del proyecto.


\subsection{Motivación para desarrollar el sistema}
La empresa ha decidido reemplazar parte de su sistema de software por un sistema nuevo, debido a la acumulación de fallas, bugs, código difícil de mantener y extender, y problemas de rendimiento en su sistema original.

Además, necesita un sistema más eficiente que pueda acomodar a su creciente número de usuarios, y que el mismo tenga la capacidad de proveer a los restaurantes, estadísticas acerca de hábitos de consumo, productos más vendidos, zonas y horarios más demandados, etc.

Los principales motivaciones para el desarrollo del sistema son:

\begin{enumerate}
    \item Mostrar a los clientes de manera sencilla e intuitiva los restaurantes asociados a PedidosNow disponibles en su zona, junto con sus productos ofrecidos.
    \item Facilitar al cliente lo más posible la exploración de la oferta de productos, su selección, pago, y seguimiento del envío.
    \item Implementar un sistema que sea eficiente a nivel computacional.
    \item Conectar de forma eficiente a clientes, repartidores y restaurantes, de tal manera de maximizar la eficiencia del sistema y evitar el desaprovechamiento de recursos, como el tiempo de los repartidores. El sistema también debe minimizar lo más posible el tiempo de envío.
    \item Generación de reportes para los restaurantes, que contengan información acerca de sus productos más vendidos, niveles de stock, perfil de los clientes más interesados en sus productos, etc.
\end{enumerate}

\subsection{Descripción de usuarios del sistema}
El sistema está orientado principalmente a tres tipos de usuarios:

\begin{enumerate}
    \item \textbf{Clientes}
    
    Personas que realizan pedidos desde sus casas u oficinas. Buscan rapidez, variedad de opciones y una interfaz intuitiva. Generalmente utilizan la aplicación desde el celular. Desean que la experiencia de búsqueda de restaurantes, productos, selección de estos últimos, el pago, y el seguimiento del envío sean sencillos y fáciles de realizar, además de que el sistema funcione con confiabilidad y seguridad.

    \item \textbf{Repartidores}
    
    Personas encargadas de recoger los pedidos en el restaurante y llevarlos hasta el cliente. Requieren información precisa sobre la dirección, el tiempo estimado, contacto con el cliente si es necesario, comisión a recibir por realizar el envío, etc. Además, tienen que poder elegir sus horarios y zonas de trabajo.

    \item \textbf{Restaurantes}
    
    Comercios gastronómicos que ofrecen sus productos a través de la plataforma. Necesitan poder gestionar sus sucursales, menú, recibir pedidos de forma clara y notificar cuando el pedido esté listo para ser retirado. Buscan confiabilidad y seguridad en el sistema. También desean recibir estadísticas acerca de sus productos más vendidos, perfil de sus clientes, etc.
\end{enumerate}

\subsection{Requisitos funcionales del sistema}

A continuación se listan los requisitos funcionales con los que debe cumplir el sistema:

\begin{enumerate}
    \item Registro y autenticación de usuarios (clientes, repartidores, restaurantes).

    \item Visualización de restaurantes asociados, junto con sus productos, según ubicación del cliente, horarios de atención, y disponibilidad de stock.
    
    \item Permitir al usuario cargar, modificar y eliminar múltiples direcciones de envío.
    
    \item Navegación por menús, categorías y productos de los restaurantes.
    
    \item Selección de productos, armado del pedido, procesamiento del pago y envío de la orden al restaurante.
    
    \item Asignación automática de pedidos a repartidores según cercanía y disponibilidad.
    
    \item Confirmación del pedido y actualización de su estado en tiempo real, mostrando detalles del pedido, sucursal y repartidor asignado.
    
    \item Sistema de cancelación de pedidos para los clientes, restaurantes y repartidores.
    
    \item Reseñas de restaurantes realizadas por clientes.
    
    \item Historial de pedidos del cliente.
    
    % \item Sistema automático de comisiones para los repartidores, que incentive a la elección de las zonas y horarios de trabajo con mayor demanda de envíos.
    
    \item Interfaz para repartidores para ver pedidos asignados, seguir rutas y confirmar entregas.
    
    \item Interfaz para repartidores para elegir su zona y horario de trabajo.
    
    \item Interfaz para restaurantes para mostrar pedidos asignados (incluyendo datos relevantes como dirección de sucursal y de envío), y notificar que están listos.
    
    \item Interfaz para los restaurantes para cargar, modificar y eliminar datos de sus sucursales, productos, etc.
    
    \item Interfaz para restaurantes para obtener reportes y estadísticas acerca de sus productos y clientes.
    
\end{enumerate}

\subsection{Requisitos no funcionales del sistema}

El sistema debe cumplir con los siguientes requisitos no funcionales:

\begin{enumerate}
\item \textbf{Usabilidad:} La interfaz de usuario (para los tres tipos de usuario) debe ser intuitiva y fácil de utilizar. También debe mostrarse correctamente en todo tipo de dispositivo, independientemente del tamaño de la pantalla.

\item \textbf{Rendimiento:} El sistema debe funcionar correctamente incluso con muchos usuarios conectados al mismo tiempo. Las operaciones deben ser eficientes y no realizar cálculos innecesarios.

\item \textbf{Disponibilidad:} El servicio debe estar disponible las 24 hs., particularmente durante los horarios de mayor demanda. Una falla podría costar a la empresa y a sus socios comerciales (los restaurantes), enormes cantidades de dinero, pérdida de confiabilidad y daño a la percepción pública de la marca.

\item \textbf{Seguridad:} Protección de datos personales y bancarios mediante cifrado y buenas prácticas.

\item \textbf{Mantenibilidad:} Código claro y modular para facilitar futuras mejoras, como el desarrollo de nuevas funcionalidades.
\end{enumerate}

\subsection{Límites y alcances del sistema}

Se reutilizarán algunas partes del sistema original, por lo que no se implementarán todas sus partes. Por esta razón, se implementará el modelo de datos que sería necesario para las características mencionadas en la sección ``Requisitos funcionales del sistema''. En cuánto a la interfaz gráfica y el backend necesario, solo se implementarán las siguientes funcionalidades:

\begin{enumerate}
    \item Sistema de login y registro para restaurantes.
    \item ABM de sucursales de los restaurantes, con autenticación.
    \item ABM de menú de cada sucursal, con autenticación.
    \item Generación de reporte con resumen de ventas de cada restaurante, por sucursal.
    \item Vista pública (no requiere autenticación) de todas las sucursales cargadas y sus menús.
    \item Sistema de administración de copias de seguridad, solo autorizado para administradores. \\
\end{enumerate}

No se implementarán las siguientes funcionalidades:

\begin{enumerate}
    \item Sistema de registro de clientes y repartidores.
    \item Sistema automático de comisiones para repartidores, que incentive la elección de las zonas y horarios de trabajo con mayor demanda de envíos.
    \item Procesamiento de los pagos con tarjeta.
    \item Seguimiento en tiempo real del pedido (ubicación del repartidor).
    \item Sistema de devoluciones y reembolsos.
    \item Sistema de atención al cliente a través de la app (tanto para clientes, como para repartidores y restaurantes).
\end{enumerate}
